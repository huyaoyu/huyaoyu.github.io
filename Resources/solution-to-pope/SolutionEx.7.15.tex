\documentclass{article}

\usepackage{amsmath}

% new mathematical symbol
\newcommand{\dd}{\; \mathrm{d}}
\newcommand{\DD}{\; \mathrm{D}}
\newcommand{\enangle}[1]{\langle {#1} \rangle}
\newcommand{\parpar}[2]{\frac{\partial {#1}}{\partial {#2}}}
\newcommand{\parpard}[2]{\frac{\partial^2 {#1}}{\partial {#2}^2}}
\newcommand{\eqrefnew}[1]{Eq.~\eqref{#1}}

\title{Solution to Ex.~7.15}
\author{Yaoyu Hu}

\begin{document}
\maketitle
Note that it is assumed the following relation holds on the axis.
\begin{equation} \label{eq:uplus}
	u^+ \equiv \frac{\enangle{U}}{u_{\tau}} = \frac{1}{\kappa}\ln\left( \frac{yu_{\tau}}{\nu} \right) + B
\end{equation}
and we know that the mean velocity profile can be approximated by the logarithmic defect law (Eq.~(7.104) in the book)
\begin{equation} \label{eq:DefectLaw}
	\frac{U_0 - \enangle{U}}{u_{\tau}} = -\frac{1}{\kappa}\ln\left( \frac{y}{R} \right)
\end{equation}
and yet
\begin{equation}
	\frac{U_0 - \bar{U}}{u_{\tau}} = \frac{3}{2\kappa}
\end{equation}
follow the analysis procedures of the channel flow, if we add up \eqrefnew{eq:uplus} and \eqrefnew{eq:DefectLaw}, we get
\begin{equation} \label{eq:added}
	\frac{U_0}{u_{\tau}} = \frac{1}{\kappa}\ln\left( \frac{y}{\delta_{\nu}} \right) + B
\end{equation}
here $y$ has the same meaning of $\delta$, so
\begin{equation}
	\frac{y}{\delta_{\nu}} = \frac{\delta}{\delta_{\nu}} = \text{Re}_{\tau} = \frac{u_{\tau} \delta}{\nu}
\end{equation}
make use of Eq.~(7.104) on the book, we get
\begin{equation} \label{eq:Re}
	\text{Re}_{\tau} = \frac{2\bar{U}\delta}{\nu} \frac{\sqrt{f}}{4\sqrt{2}} = \text{Re}\frac{\sqrt{f}}{4\sqrt{2}}
\end{equation}
from Eq.~(7.107) on the book, the relationship between $U_0$ and $u_{\tau}$ in Eq.~(7.107) could be written
\begin{equation} \label{eq:U0}
	U_0 = \frac{3}{2\kappa}u_{\tau} + \bar{U}
\end{equation}
by using \eqrefnew{eq:U0} and \eqrefnew{eq:Re}, \eqrefnew{eq:added} can be written as
\begin{equation} \label{eq:BeforeFinal}
	\frac{3}{2\kappa} + \frac{\bar{U}}{u_{\tau}} = \frac{1}{\kappa}\ln\left( \text{Re} \frac{\sqrt{f}}{4\sqrt{2}} \right) + B
\end{equation}
use Eq.~(7.104) again, \eqrefnew{eq:BeforeFinal} can be changed into
\begin{equation}
	\frac{1}{\sqrt{f}} = \frac{1}{2\sqrt{2}\kappa}\ln\left( \text{Re}\sqrt{f} \right) - \frac{5\ln 2 - 2\kappa B + 3}{4\sqrt{2}\kappa}
\end{equation}


\end{document}